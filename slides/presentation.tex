%% Beispiel-Präsentation
\documentclass[de]{sdqbeamer} 

%% Titelbild
\titleimage{blender-6}

%% Gruppenlogo
\grouplogo{ae} 

%% Gruppenname
\groupname{ITI Sanders | KASTEL Schreiber}

% Beginn der Präsentation

\title[SAT Seminar]{Fortgeschrittene Themen im SAT Solving}
\subtitle{Seminar Kick-Off} 
\author[Iser, Schreiber]{Ashlin Iser, Dominik Schreiber, Niccolò Rigi-Luperti}

\date[2025-11-06]{6. November 2025}

% Literatur 

%\usepackage[citestyle=authoryear,bibstyle=numeric,hyperref,backend=biber]{biblatex}
%\addbibresource{presentation.bib}
%\bibhang1em

\usepackage[absolute,overlay]{textpos}
\usepackage{tikz}
\usepackage{tabularx}
%\usepackage[texcoord,grid,gridcolor=red!10,subgridcolor=green!10,gridunit=pt]{eso-pic}

\definecolor{RedOrange}{HTML}{AA7700}
\newcommand{\highl}[1]{\textcolor{blue}{#1}}
\newcommand{\highlo}[1]{\textcolor{RedOrange}{#1}}

\usepackage{hyperref}
\hypersetup{
	colorlinks = true,
	urlcolor = {darkgray}
}

\usepackage[style=verbose,backend=biber]{biblatex}
\addbibresource{seminar.bib}

\begin{document}
	
	%Titelseite
	\KITtitleframe
	
	\begin{frame}{Organisatorisches}
		\begin{itemize}
			\item Ausführlicher Vortrag (30 min. + 15 min. Fragen), keine Ausarbeitung
			\item 2-3 Papiere aus einem gemeinsamen Themenblock einordnen, kommentieren, vergleichen
			\item Unterstützung und Beratung von Betreuerseite
			\item $n$ Vortragstermine (für kleine $n$) nach Absprache
		\end{itemize}
		
		\ 
		
		\begin{columns}
			\begin{column}{0.33\textwidth}
				\centering \includegraphics[height=3cm]{iser.jpg}\\
				iser@kit.edu\\
			\end{column}%
			\begin{column}{0.33\textwidth}
				\centering \includegraphics[height=3cm]{nicco.jpg}\\
				niccolo.rigi-luperti@kit.edu\\
			\end{column}%
			\begin{column}{0.33\textwidth}
				\centering \includegraphics[height=3cm]{schreiber.png}\\
				schreiber@kit.edu
			\end{column}
		\end{columns}
	\end{frame}
	
	\begin{frame}{Topics: The Big Picture}
		\begin{itemize}
			\item Which \highl{innovative paradigms} for SAT solving and its extensions have emerged in the past few years?
			\item How can we exploit \highl{modern hardware} for \highl{dependable SAT solving} and its applications?
			\item How can we \highl{understand and exploit statistical properties} of solvers and problems?
		\end{itemize}
	\end{frame}
	
	\begin{frame}{SAT and its Extensions}
		\begin{exampleblock}{1. CDCL and Local Search}
			\href{https://doi.org/10.1007/978-3-030-80223-3\_6}{2021, Cai et al., “Deep Cooperation of CDCL and local search for SAT”}\\
			\href{https://doi.org/10.1613/jair.1.13666}{2022, Cai et al., “Better decision heuristics in CDCL through local search and target phases”}\\
		\end{exampleblock}
		\vspace*{-2mm}
		\begin{exampleblock}{2. Variable Addition}
			\href{https://link.springer.com/chapter/10.1007/978-3-642-39611-3_14}{2012, Manthey et al.: Automated Reencoding of Boolean Formulas}\\
			\href{https://arxiv.org/abs/2307.01904}{2023, Haberlandt et al.: Effective Auxiliary Variables via Structured Reencoding}\\
		\end{exampleblock}
		\vspace*{-2mm}
		\begin{exampleblock}{3. MaxSAT}
			\href{https://doi.org/10.1007/978-3-319-40970-2\_34}{2016, Saikko et al., “LMHS: A SAT-IP Hybrid MaxSAT Solver”}\\
			\href{https://doi.org/10.24963/ijcai.2023/215}{2023, Ihalainen et al., “Unifying Core-Guided and Implicit Hitting Set Based Optimization”}\\
			\href{https://www.jair.org/index.php/jair/article/download/15986/27058}{2025, Ihalainen et al., “Unifying SAT-Based Approaches to Maximum Satisfiability Solving”}
		\end{exampleblock}
		\vspace*{-2mm}
		\begin{exampleblock}{4. Pseudo-Boolean Reasoning: Propagation beyond Clauses}
			% \href{https://doi.org/10.1609/aaai.v35i5.16494}{2021, Gocht and Nordström, “Certifying Parity Reasoning Efficiently Using Pseudo-Boolean Proofs”}\\
			\href{https://doi.org/10.4230/LIPIcs.SAT.2024.22}{2024, Nieuwenhuis et al., “Speeding-up Pseudo-Boolean Propagation”}\\
			\href{https://ceur-ws.org/Vol-4008/POS_paper09.pdf}{2025, Müßig \& Johannsen, “Improving Watched Pseudo-Boolean Propagation with Significant Literals”}
		\end{exampleblock}
	\end{frame}
	
	\iffalse
	\begin{frame}{Preprocessing beyond Resolution}
		\begin{exampleblock}{4. Extended Resolution}
			2010, Audemard et al., “A Restriction of Extended Resolution for Clause Learning SAT Solvers”\\
			2023, Haberlandt et al., “Effective Auxiliary Variables via Structured Reencoding”
		\end{exampleblock}
		\begin{exampleblock}{5. Propagation Redundancy}
			2023, Reeves et al., “Preprocessing of Propagation Redundant Clauses”\\
			2023, Gao, “Kissat MAB prop in SAT Competition 2023”
		\end{exampleblock}
	\end{frame}
	\fi
	
	\begin{frame}{Dependable Solving and Modern Hardware}
		\begin{exampleblock}{5. Proofs in Parallel \& Distributed SAT}
			\href{https://doi.org/10.1007/978-3-319-40970-2\_15}{2016, Heule et al., “Solving and verifying the boolean pythagorean triples problem via cube-and-conquer”}\\
			\href{https://doi.org/10.1007/978-3-031-30823-9\_18}{2023, Michaelson et al., “Unsatisfiability Proofs for Distributed Solvers”}\\
			\href{https://doi.org/10.4230/LIPIcs.SAT.2024.25}{2024, Schreiber, “Trusted Scalable SAT Solving with on-the-fly LRAT Checking”}
		\end{exampleblock}
		\begin{exampleblock}{6. GPU-accelerated SAT Solving}
			\href{https://doi.org/10.1007/978-3-030-17462-0\_2}{2019, Osama and Wijs, “Parallel SAT Simplification on GPU Architectures”}\\
			\href{https://www.cs.toronto.edu/~meel/Papers/sat21-psm.pdf}{2021, Prevot et al., “Leveraging GPUs for Effective Clause Sharing in Parallel SAT Solving”}\\
			\href{https://doi.org/10.1007/s10703-023-00432-z}{2024, Osama et al., “SAT Solving with GPU Accelerated Inprocessing”}\\
			\href{https://ojs.aaai.org/index.php/AAAI/article/download/33211/35366}{2025, Cen et al., ``Massively parallel continuous local search for hybrid SAT solving on GPUs''}
		\end{exampleblock}
		\begin{exampleblock}{7. Parallel Model Counting}
			\href{https://link.springer.com/chapter/10.1007/978-3-319-24318-4_5}{Burchard et al. (2015): Laissez-Faire Caching for Parallel \#SAT Solving}
			\\
			\href{https://univ-artois.hal.science/hal-03300776/document}{Lagniez et al. (2018): DMC: A Distributed Model Counter}
		\end{exampleblock}
	\end{frame}
	
	\begin{frame}{Data Science and Applications}
		\begin{exampleblock}{8. Algorithm Selection for NP Problems}
			\href{https://ada.liacs.nl/papers/XuEtAl08.pdf}{2008, Xu et al., “SATzilla: Portfolio-based Algorithm Selection for SAT”}\\
			\href{https://ada.liacs.nl/papers/HeiEtAl23.pdf}{2022, Heins et al., “A study on the effects of normalized TSP features for automated algorithm selection”}
		\end{exampleblock}
		\vspace*{-2mm}
		\begin{exampleblock}{9. Circuit Equivalence Checking}
			\href{https://dl.acm.org/doi/10.5555/777092.777187}{Bacchus et al. (2002): Enhancing Davis Putnam with Extended Binary Clause Reasoning}\\
			\href{https://drops.dagstuhl.de/entities/document/10.4230/LIPIcs.SAT.2024.6}{Biere et al. (2024): Clausal Congruence Closure}
		\end{exampleblock}
		\vspace*{-2mm}
		\begin{exampleblock}{10. Future Application: Quantum Circuit Layout}
			\href{https://doi.org/10.4230/LIPIcs.SAT.2024.26}{2024, Shaik et al., “Optimal Layout Synthesis for Deep Quantum Circuits on NISQ Processors with 100+ Qubits”}\\
			\href{https://doi.org/10.4230/LIPIcs.SAT.2024.29}{2024, Yang et al., “Quantum Circuit Mapping Based on Incremental and Parallel SAT Solving”}
		\end{exampleblock}
		\vspace*{-2mm}
		\begin{exampleblock}{11. Formally Explainable AI (FXAI)}
			\href{https://arxiv.org/pdf/2410.14219}{2024, Paul et al., “Formal Explanations for Neuro-Symbolic AI”}\\
			\href{https://arxiv.org/abs/2406.11873}{2024, Marques-Silva, “Logic-Based Explainability: Past, Present \& Future”}
		\end{exampleblock}
	\end{frame}
	
	
	
	
	
\end{document}
